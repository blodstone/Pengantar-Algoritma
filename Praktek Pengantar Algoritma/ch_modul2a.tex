%\chapter{Perulangan dan Array}
%
%%\section{Petunjuk}
%%\begin{itemize}
	%%\item Perhatikan petunjuk Dosen mengenai beda Latihan dan Permasalahan !
	%%\
	%%\item Perhatikan dan ikuti petunjuk dosen mengenai bagaimana cara menggunakan program Judge untuk mengevaluasi permasalahan yang Anda kerjakan!
%%\end{itemize}
%%
%%\section{Latihan}
%%
%%
%%\begin{latihan}
%%Ketikkan dan jalankan listing \ref{lst:anakAyam1} dan \ref{lst:anakAyam2} berikut pada dua file berbeda ! 
%%\lstinputlisting[language=Python,
									%%label={lst:anakAyam1},
									%%caption=Lagu Anak Ayam (for)]									
 							  %%{code/AnakAyam.py}
%%
%%
%%
%%\lstinputlisting[language=Python,
									%%label={lst:anakAyam2},
									%%caption=Lagu Anak ayam (while)]									
 							  %%{code/AnakAyam2.py}
								%%
%%Bandingkan kedua hasil listing diatas ! 
%%\end{latihan}
%%
%%\begin{latihan}
%%Ketikkan dan jalankan listing \ref{lst:polaAngka1}  berikut  ! 
%%\lstinputlisting[language=Python,
									%%label={lst:polaAngka1},
									%%caption=Lagu Anak Ayam (for)]									
 							  %%{code/PolaAngka1.py}
%%Dapatkah Anda menerangkan apa tujuan algoritma diatas ?  
%%\end{latihan}
%%
%%\begin{latihan}
%%Ketikkan dan jalankan listing \ref{lst:polaBintang1}  berikut  ! 
%%\lstinputlisting[language=Python,
									%%label={lst:polaBintang1},
									%%caption=Pola Bintang (for bersarang)]									
 							  %%{code/PolaBintang1.py}
%%Dapatkah Anda menerangkan apa tujuan algoritma diatas ?  
%%
%%\end{latihan}
%%
%%Ubah seluruh listing diatas menjadi bentuk kompetisi pemrograman, submit masing - masing program secara berurutan pada Problem 3-1-Latihan, Problem 3-2-Latihan, Problem 3-3-Latihan, dan Problem 3-4-Latihan
%
%\newpage
\section{Permasalahan}
\begin{permasalahan}{Permasalahan Bilangan Kelipatan}\\
\label{prob:bilanganKelipatan}
	Hasilkan serangkaian $n$ bilangan yang merupakan kelipatan dari angka $m$ yang dimasukkan.\\\\
	\textbf{Masukan}\\
	Satu baris masukkan yang terdiri dari $n$ dan $m$. $n$ merupakan panjang rangkaian yang akan dihasilkan sedangkan $m$ adalah kelipatan bilangan yang diinginkan. $n$ dan $m$ adalah bilangan bulat\\
	\textbf{Keluaran}\\
	Satu set rangkaian bilangan bulat dengan panjang rangkaian $n$ dan merupakan kelipatan dari $m$. Semua rangkaian bermula dari angka $m$.\\
	\begin{center}
	\textbf{Test Case 1}\\
	\end{center}
	\textbf{Masukan}\\
	6 5\\
	\textbf{Keluaran}\\
	5 10 15 20 25 30 \\
	\begin{center}
	\textbf{Test Case 2}\\
	\end{center}
	\textbf{Masukan}\\
	3 4\\
	\textbf{Keluaran}\\
	4 8 12 \\
	\begin{center}
	\textbf{Test Case 3}\\
	\end{center}
	\textbf{Masukan}\\
	10 2\\
	\textbf{Keluaran}\\
	2 4 6 8 10 12 14 16 18 20\\
\end{permasalahan}

\newpage
\begin{permasalahan}{Palindrom}\\
\label{prob:Palindrom}
	 Palindrom adalah kata yang dibaca dari depan atau belakang dengan cara yang sama. Dalam bahasa Inggris, kata ``a`` dan ``I`` adalah palindrom yang paling sederhana.``Racecar`` dan ``Hannah`` adalah yang paling ilustrative. Dalam bahasa Indonesia misalnya ``ada`` dan ``kayak``. Dapatkah kamu menentukan sebuah kata palindrom atau tidak ?\\\\
	\textbf{Masukan}\\
	Sebuah string terdiri dari satu kata dengan case yang bervariasi. \\
	\textbf{Keluaran}\\
	palindrom atau bukan palindrom\\
	\begin{center}
	\textbf{Test Case 1}\\
	\end{center}
	\textbf{Masukan}\\
	budidaya\\
	\textbf{Keluaran}\\
	bukan palindrom\\
	
	\begin{center}
	\textbf{Test Case 2}\\
	\end{center}
	\textbf{Masukan}\\
	rotavator\\
	\textbf{Keluaran}\\
	palindrom\\
	
	\textit{Catatan :}\\
	\begin{enumerate}
	\item \textit{Proses Reverse harus menggunakan perulangan ! Tidak boleh menggunakan shortcut array !}
	\item \textit{NB : Walau melibatkan percabangan, percabangan yang dilibatkan di luar perulangan.}
	\end{enumerate}
\end{permasalahan}

\newpage
\begin{permasalahan}{Jumlah dan Rata I}\\
\label{prob:BanyakDiriku}
	 Tentukan Jumlah dan Rata dari masukkan\\\\
	\textbf{Masukan}\\
	(banyak angka) 1 buah bilangan bulat yang menentukan berapa banyak angka \\
	Kumpulan bilangan bulat acak sebanyak (banyak angka) \\
	\textbf{Keluaran}\\
	Format : (Jumlah) (Rata)\\
	(Jumlah) dalam bilangan bulat\\
	(Rata)  dalam bilangan pecahan 3 dibelakang koma\\
	\begin{center}
	\textbf{Test Case}\\
	\end{center}
	\textbf{Masukan}\\
	5\\
	100\\
	100\\
	100\\
	100\\
	100\\
	\textbf{Keluaran} \\
		500 100.000
\end{permasalahan}


\newpage
\begin{permasalahan}{Jumlah dan Rata II}\\
\label{prob:BanyakDiriku2}
	  Tentukan Jumlah dan Rata dari masukkan\\\\
	\textbf{Masukan}\\
	Kumpulan bilangan bulat acak \\
	\textbf{Keluaran}\\
	Format : (Jumlah) (Rata)\\
	(Jumlah) dalam bilangan bulat\\
	(Rata)  dalam bilangan pecahan 3 dibelakang koma
	\begin{center}
	\textbf{Test Case}\\
	\end{center}
	\textbf{Masukan}\\
	100 100 100 100 100 \\
	\textbf{Keluaran}\\
		500 100.000\\
\end{permasalahan}


\newpage
\begin{permasalahan}{Konversi}\\
\label{prob:Konversi}
	 Dapatkah kamu mengkonversi desimal ke biner, oktal dan hexa ?\\\\
	\textbf{Masukan}\\
	Sebuah Bilangan bulat positif yang mewakili desimal yang akan dikonversi
	\textbf{Keluaran}\\
	3 buah tipe data string dengan format : (biner) (oktal) (hexa) 	
	\begin{center}
	\textbf{Test Case}\\
	\end{center}
	\textbf{Masukan}\\
  65535\\
	\textbf{Keluaran}\\
		1111111111111111 177777 FFFF 
\end{permasalahan}


\newpage
\begin{permasalahan}{Berapakah Banyak Diriku ? }\\
\label{prob:BanyakDiriku}
	 Dapatkah kamu menentukan ada berapa banyak angka 0 sampai 9 dari masukkan ?\\\\
	\textbf{Masukan}\\
	(banyak angka) 1 buah bilangan bulat positif yang menentukan berapa banyak angka.\\
	Kumpulan bilangan acak dari 0 sampai 9 sebanyak (banyak angka) \\
	\textbf{Keluaran}\\
	(banyak angka 0)\\(banyak angka 1)\\(banyak angka 2)\\...\\(banyak angka 9)\\secara berurutan dan dalam bilangan bulat positif
	\begin{center}
	\textbf{Test Case}\\
	\end{center}
	\textbf{Masukan}\\
	15\\
	3\\
	8\\
	2\\
	0\\
	4\\
	5\\
	6\\
	9\\
	1\\
	7\\
	3\\
	3\\
	1\\
	2\\
	2\\
	\textbf{Keluaran}\\
		1\\2\\3\\3\\1\\1\\1\\1\\1\\1
\end{permasalahan}


\newpage
\begin{permasalahan}{Berapakah Banyak Diriku ? (Versi II) }\\
\label{prob:BanyakDiriku2}
	 Dapatkah kamu menentukan ada berapa banyak angka 0 sampai 9 dari masukkan ?\\\\
	\textbf{Masukan}\\
	Kumpulan bilangan acak dari 0 sampai 9 dengan banyak tertentu\\
	\textbf{Keluaran}\\
	(banyak angka 0)\\(banyak angka 1)\\(banyak angka 2)\\...\\(banyak angka 9)\\secara berurutan dan dalam bilangan bulat positif
	\begin{center}
	\textbf{Test Case}\\
	\end{center}
	\textbf{Masukan}\\
	3 
	8  
	2 
	0 
	4 
	5 
	6 
	9 
	1 
	7 
	3 
	3 
	1 
	2 
	2 \\
	\textbf{Keluaran}\\
		1\\2\\3\\3\\1\\1\\1\\1\\1\\1
\end{permasalahan}



%\begin{permasalahan}
%Cetak Pola Bintang
%\end{permasalahan}
%
%\begin{permasalahan}

%Cetak Pola Deret
%\end{permasalahan}
%
%\begin{permasalahan}

%Diberikan String
%Ambil Huruf Awal, Tengah dan Akhir
%\end{permasalahan}
%
%\begin{permasalahan}
%Huruf - Huruf apa saja yang ada dari kalimat ? Musti Distinct
%\end{permasasalahan}
%
%\begin{permasalahan}
%Ada berapa angka yang dicari dari kumpulan angka yang berdempetan satu sama lain
%\end{permasasalahan}
%
%\begin{permasalahan}
%Ada berapa angka yang dicari dari kumpulan angka yang berdempetan satu sama lain
%\end{permasasalahan}
%
%\begin{permasalahan}
%
%\end{permasasalahan}
%
%\begin{permasalahan}
%double_char('The') → 'TThhee'
%double_char('AAbb') → 'AAAAbbbb'
%double_char('Hi-There') → 'HHii--TThheerree'
%\end{permasasalahan}


%\begin{panduan}{Tes}
%\begin{enumerate}
	%\item Baca Permasalahan \ref{prob:bilanganKelipatan}.
	%\item Buka Pyscripter.
	%\item Ketikkan Listing \ref{lst:permasalahan1}.
	%\begin{listprog}{Permasalahan 1 (permasalahan1.py)}
		%\label{lst:permasalahan1}
		%\begin{lstlisting}[language=Python]
		%n = input()
		%m = input()
		%for i in range(1,n+1):
	    %print m*i,
		%\end{lstlisting}
	%\end{listprog}
	%\item \textit{Save} file tersebut sebagai permasalahan1.py
	%\item Masuk ke http://elearning.mikroskil.ac.id, dan masuk ke \textit{Course} Pengantar Algoritma.
	%\item Klik di \textbf{Tugas Praktek 1: Permasalahan Bilangan Kelipatan}.
	%\item Klik Browse.
	%\item Pilih file permasalahan1.py yang sudah anda \textit{save} dan klik ok.
	%\item Refresh Browser sampai tulisan \textbf{Status} dari Pending menjadi Accepted. Jika ada error berarti ada kesalahan. Cek kembali Listing \ref{lst:permasalahan1}.
%\end{enumerate}
%\end{panduan}


%
%\newpage
%\begin{permasalahan}{Permasalahan Penjumlahan Bilangan Ganjil}\\
	%Diberikan sebuah bilangan $n$ carilah jumlah dari semua bilangan ganjil antara 0 sampai dengan bilangan $n$ tersebut.\\
	%\textbf{Masukan}\\
	%Sebuah bilangan bulat $n$.\\
	%\textbf{Keluaran}\\
	%Sebuah bilangan bulat $z$ dimana $z$ merupakan hasil penjumlahan dari semua bilangan ganjil yang ada antara 0 sampai bilangan $n$ tersebut (bilangan $n$ termasuk).\\
	%\begin{center}
	%\textbf{Test Case 1}\\
	%\end{center}
	%\textbf{Masukan}\\
	%20\\
	%\textbf{Keluaran}\\
	%100\\
	%\textit{Penjelasan: jumlah bilangan ganjil antara 1 sampai dengan 20 \\adalah 1+3+5+7+9+11+13+49=100}\\
	%\begin{center}
	%\textbf{Test Case 2}\\
	%\end{center}
	%\textbf{Masukan}\\
	%50\\
	%\textbf{Keluaran}\\
	%625\\
	%\begin{center}
	%\textbf{Test Case 3}\\
	%\end{center}
	%\textbf{Masukan}\\
	%90\\
	%\textbf{Keluaran}\\
	%2025\\
%\end{permasalahan}
%

