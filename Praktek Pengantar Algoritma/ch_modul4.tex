\chapter{Fungsi}

\begin{permasalahan}{Kalkulator Sederhana}\\
	Hasilkan sebuah program untuk dapat melakukan perhitungan operasi aritmetika yang mencakup penjumlahan, pengurangan, perkalian dan pembagian.\\
	\textbf{Masukan}\\
	Masukan terdiri dari tiga baris. Baris pertama merupakan operand (bilangan yang dioperasikan) pertama. Baris kedua merupakan operand kedua. Baris ketiga merupakan operator aritmetika dari kedua operand tersebut dimana :
	\begin{enumerate}
		\item operator '+' untuk penjumlahan.
		\item operator '-' untuk pengurangan.
		\item operator '*' untuk perkalian.
		\item operator '/' untuk pembagian.
	\end{enumerate}
	\textbf{Keluaran}\\
	Hasil dari operasi aritmetika tersebut.\\
	\begin{center}
	\textbf{Test Case 1}\\
	\end{center}
	\textbf{Masukan}\\
	6\\
	4\\
	$-$\\
	\textbf{Keluaran}\\
	2\\
	\begin{center}
	\textbf{Test Case 2}\\
	\end{center}
	\textbf{Masukan}\\
	95\\
	8\\
	$/$\\
	\textbf{Keluaran}\\
	11.875\\
\end{permasalahan}

\newpage
\begin{permasalahan}{Kalkulator Sederhana Multi Operasi}\\
	Kembangkan program sebelumnya sehingga dapat melakukan perhitungan operasi aritmetika lebih dari satu kali.\\
	\textbf{Masukan}\\
	Baris pertama merupakan jumlah operasi aritmetika. Baris kedua merupakan operand (bilangan yang dioperasikan) pertama. Baris ketiga merupakan operand kedua. Baris keempat merupakan operator aritmetika untuk kedua operand tersebut.\\
	Untuk tiap dua baris berikutnya, baris pertama merupakan operand baru dan baris kedua merupakan operator aritmetika untuk melakukan operasi aritmetika terhadap operand baru dan hasil dari operasi sebelumnya.
	\textbf{Keluaran}\\
	Hasil terakhir dari seluruh operasi aritmetika tersebut.\\
	\begin{center}
	\textbf{Test Case 1}\\
	\end{center}
	\textbf{Masukan}\\
	2\\
	6\\
	4\\
	$-$\\
	3\\
	$+$\\
	\textbf{Keluaran}\\
	5\\
	Deskripsi : 6-4+3=5 (Jumlah operasi 2).
	\begin{center}
	\textbf{Test Case 2}\\
	\end{center}
	\textbf{Masukan}\\
	3\\
	8\\
	2\\
	$/$\\
	5\\
	$+$\\
	7\\
	$*$\\
	\textbf{Keluaran}\\
	63\\
\end{permasalahan}

