\chapter{Ad Hoc 1}

\begin{permasalahan}{Pencetakan Alphabet}\\
Pencetakan alphabet dilakukan berdasarkan String yang diberikan yang berisi numerik (jumlah alphabet yang akan dicetak) dan alphabet yang akan dicetak.\\
	\\
	\textbf{Masukan}\\
	Sebuah String yang elemennya diselang-selingi oleh alphabet dan numerik. Numerik menunjukkan banyaknya alphabet yang akan dicetak (1 <= Numerik <= 9).  Alphabet disamping numerik tersebut merupakan alphabet yang akan dicetak.\\
	\textbf{Keluaran}\\
	Sebuah String yang elemennya berupa kumpulan alphabet.\\
	\begin{center}
	\textbf{Test Case 1}\\
	\end{center}
	\textbf{Masukan}\\
	2f3d7e\\
	\textbf{Keluaran}\\
	ffdddeeeeeee\\
	\begin{center}
	\textbf{Test Case 2}\\
	\end{center}
	\textbf{Masukan}\\
	5r4f1u\\
	\textbf{Keluaran}\\
	rrrrrffffu\\
	\begin{center}
	\textbf{Test Case 3}\\
	\end{center}
	\textbf{Masukan}\\
	4u6t\\
	\textbf{Keluaran}\\
	uuuutttttt\\
\end{permasalahan}

\newpage

\begin{permasalahan}{Pencetakan Alphabet}\\
Pembalikan String dilakukan dengan menukar posisi karakter pertama dengan karakter terakhir, karakter kedua dengan karakter kedua terakhir dan seterusnya.\\
	\\
	\textbf{Masukan}\\
	Sebuah String yang akan dibalik.\\
	\textbf{Keluaran}\\
	Sebuah String yang telah dibalik.\\
	\begin{center}
	\textbf{Test Case 1}\\
	\end{center}
	\textbf{Masukan}\\
	Mikroskil\\
	\textbf{Keluaran}\\
	liksorkiM\\
	\begin{center}
	\textbf{Test Case 2}\\
	\end{center}
	\textbf{Masukan}\\
	permasalahan 123\\
	\textbf{Keluaran}\\
	321 nahalasamrep\\
	\begin{center}
	\textbf{Test Case 3}\\
	\end{center}
	\textbf{Masukan}\\
	abc\\
	\textbf{Keluaran}\\
	cba\\
\end{permasalahan}

\newpage

