\chapter{Divide and Conguer}

\begin{permasalahan}{Pencarian bilangan}\\
Diberikan sebuah rangkaian angka, tentukan apakah sebuah angka `x' terdapat di dalam rangkaian tersebut dengan menggunakan \textit{binary sort}.\\
	\\
	\textbf{Masukan}\\
	Terdapat dua baris. Baris pertama, sebuah set angka bilangan bulat. Baris kedua, bilangan `x' yang hendak dicari.\\
	\textbf{Keluaran}\\
	Angka -1 jika tidak ditemukan, lokasi index dari bilangan tersebut jika ditemukan.\\
	\begin{center}
	\textbf{Test Case 1}\\
	\end{center}
	\textbf{Masukan}\\
	1 12 4 13 15 3 5 7\\
	4\\
	\textbf{Keluaran}\\
	3\\
	\begin{center}
	\textbf{Test Case 2}\\
	\end{center}
	\textbf{Masukan}\\
	4 10 5\\
	6\\

	\textbf{Keluaran}\\
  -1\\
	\begin{center}
	\textbf{Test Case 3}\\
	\end{center}
	\textbf{Masukan}\\
	10 23 14\\
	23\\
	\textbf{Keluaran}\\
  3\\
\end{permasalahan}

\newpage

\begin{permasalahan}{Simpanse Playboy}\\
Seekor simpanse playboy, kita panggil namanya Luchu, sedang memilih pasangannya, simpanse-simpanse betina yang sedang berbaris. Luchu sendiri dalam memilih simpanse betina memiliki kriteria-kriteria tertentu. Luchu menyukai simpanse betina yang memiliki tinggi badan yang lebih pendek darinya, tapi juga bisa menerima simpanse betina yang memiliki tinggi badan sedikit lebih tinggi darinya. Luchu akan menolak simpanse betina yang memiliki tinggi badan yang sama tinggi seperti dirinya. Tugas anda adalah membantu Luchu dalam memilih simpanse-simpanse betina yang sesuai kriterianya.\\
	\\
	\textbf{Masukan}\\
	Terdapat empat baris untuk masukan ini.\\
	- Baris pertama menandakan jumlah simpanse betina yang akan dipilih oleh Luchu.\\
	- Baris kedua menandakan sejumlah tinggi badan simpanse betina tersebut (yang tidak disortir), kita andaikan tinggi badan simpanse-simpanse betina tersebut unik (tidak ada dua simpanse yang tinggi badannya sama).\\
	- Baris ketiga adalah jumlah tinggi badan Luchu (kita misalkan seandainya Luchu memiliki tinggi badan yang berbeda-beda.)\\
	- Baris terakhir menandakan sejumlah tinggi badan Luchu.\\
	\textbf{Keluaran}\\
	Keluarannya berupa sejumlah baris sesuai dengan jumlah tinggi badan Luchu. Setiap baris terdiri dari dua angka dimana angka pertama menanda simpanse betina dengan tinggi badan paling tinggi tapi tidak lebih tinggi atau sama dengan Luchu, sedangkan angka kedua menandakan simpanse betina dengan tinggi badan paling rendah tapi lebih tinggi dari Luchu.\\
	Jika tidak ada simpanse betina yang cocok maka berikan tanda `x'\\
	\begin{center}
	\textbf{Test Case 1}\\
	\end{center}
	\textbf{Masukan}\\
	4\\
	7 5 4 1\\
	4\\
	4 6 8 10\\
	\textbf{Keluaran}\\
	1 5\\
	5 7\\
	7 X\\
	7 X\\
	\begin{center}
	\textbf{Test Case 2}\\
	\end{center}
	\textbf{Masukan}\\
	5\\
  9 8 2 1 3\\
	3\\
	1 7 10\\ 
	\textbf{Keluaran}\\
  x 2\\
  3 8\\
  9 x\\
\end{permasalahan}