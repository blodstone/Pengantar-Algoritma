\chapter{Fungsi (\textit{Function})}

\section{Apa itu Fungsi?}
Kita mengenal fungsi melalui pelajaran Matematika seperti misalnya:
$g(x) = x^4/4 - x^3/3 - 3x^2$

$g(x)$ merupakan sebuah fungsi dengan $x$ adalah parameternya. Untuk memanggil fungsi tersebut, kita menggantikan parameter $x$ menjadi nilai yang diinginkan, misalnya $g(5)$ sehingga:
$g(5) = 5^4/4 - 5^3/3 - 3*5^2$
$g(5) = 39.58333$

Kita bisa menggantikan apapun nilai parameter $x$ dan akan mendapatkan nilai $g(x)$ yang berbeda-beda.

Di dalam Teknik Informatika, fungsi atau \textit{function} merupakan bagian dari algoritma yang dikhususkan untuk mengeksekusi perintah tertentu. Fungsi ditujukan untuk menggantikan perintah yang berulang di suatu program atau yang sering dipanggil. Dengan fungsi algoritma menjadi lebih terstruktur, dinamis, dan mudah diperbaiki.

\section{Deklarasi Fungsi}
Fungsi bersifat mandiri dimana fungsi memiliki nama, variabel, dan kumpulan perintah sendiri. Fungsi dapat menerima nilai dari pemanggil fungsi dan dapat mengembalikan nilai ke pemanggil tersebut dan keduanya tersebut tidak bersifat wajib. Struktur fungsi adalah sebagai berikut.

\begin{tabbing}
\textbf{function} function name (variable1, variable2)~~~~\=\#Deklarasi nama dan variabel fungsi\\
~~~~~statements\\
~~~~~\textbf{return} value\>\#Opsional, untuk mengembalikan nilai ke pemanggil fungsi\\
\textbf{end function}
\end{tabbing}

Format penulisan \textbf{fungsi} dalam Bahasa Python adalah sebagai berikut.

\begin{tabbing}
\textbf{def} $<$function\_name$>$ (variable1, variable2):\\
~~~~~statements\\
~~~~~\textbf{return} value
\end{tabbing}

\subsection{Pemanggil Fungsi}
Untuk dapat menggunakan fungsi, diperlukan pemanggil fungsi. Pemanggil fungsi harus menggunakan nama yang sama dengan fungsi yang dipanggil dan jumlah parameternya juga harus sama. Parameter pemanggil fungsi dapat berupa variabel ataupun konstanta. Pemanggil fungsi dapat berisi sebuah nilai dimana nilai tersebut berasal dari nilai yang dikembalikan oleh fungsi tersebut. Contoh penulisan pemanggil fungsi pada algoritma berikut ini.

\begin{tabbing}
$<$function\_name$>$ (variable1, constant1)~~~~\=\#Pemanggil fungsi dapat berisi nilai\\
\end{tabbing}

Contoh-contoh berikut akan menunjukkan penggunaan fungsi pada bahasa python.

\begin{listprog}{fungsiPangkatDua.py}
	\label{lst:fungsiPangkatDua}
	\begin{lstlisting}[language=Python]
	def pangkatDua(x):
			return x * x
	c = pangkatDua(6) #Variabel c akan bernilai 36
	print c
	\end{lstlisting}
\end{listprog}

\begin{listprog}{fungsiCekGanjil.py}
	\label{lst:fungsiCekGanjil}
	\begin{lstlisting}[language=Python]
	def cekGanjil(y):
    if (y%2 != 0):
        print 'Angka ganjil'
    else:
        print 'Bukan angka ganjil'

	cekGanjil(20) #Layar akan mencetak tulisan 'Bukan angka ganjil'
	\end{lstlisting}
\end{listprog}

\section{Variabel Global dan Variabel Lokal}

Variabel global merupakan variabel yang bersifat global (luas) dalam artian variabel tersebut dapat dipanggil, digunakan atau dimanipulasi pada baris program manapun selama masih dalam satu berkas. Variabel lokal merupakan variabel yang hanya bisa dipanggil, digunakan atau dimanipulasi pada fungsi/prosedur variabel tersebut dideklarasikan. Untuk dapat memahaminya, lihatlah contoh berikut.

\begin{listprog}{fungsiRekursifPangkat.py}
	\label{lst:fungsiRekursifPangkat}
	\begin{lstlisting}[language=Python]
	vari = 0 #Variabel global

	def cetak1():
    global vari
    vari2 = 4 #Variabel lokal
    print vari #Isi vari dicetak
    print vari2 #Isi vari2 dicetak

	def cetak2():
    global vari
    print vari #Isi vari dicetak
    print vari2 #Terjadi kesalahan karena vari2 tidak dikenal

	cetak1()
	cetak2()
	\end{lstlisting}
\end{listprog}

\section{Rekursif}

Rekursif merupakan jenis fungsi yang dapat memanggil dirinya sendiri \marginnote{Fungsi yang tidak rekursif disebut fungsi iteratif}. Fungsi rekursif mendeklarasikan atau menempatkan pemanggil fungsi di dalam fungsi itu sendiri. Setiap pemanggilan fungsi rekursif akan dikembalikan satu nilai seperti pemanggilan fungsi biasa. Fungsi rekursif akan mengembalikan nilai final hasil ke pemanggil utamanya yang berada di luar fungsi. 

Contoh berikut akan menunjukkan pemakaian fungsi rekursif.

\begin{listprog}{fungsiRekursifPangkat.py}
	\label{lst:fungsiRekursifPangkat}
	\begin{lstlisting}[language=Python]
	def pangkat(x,y):
			if (y == 0):
					return 1
			elif (y == 1):
					return x
			else:
					return x*pangkat(x,y-1)

	v = pangkat(6,3) #Variabel v akan bernilai 216
	print v
	\end{lstlisting}
\end{listprog}

Penggunaan fungsi rekursif ditujukan untuk menyelesaikan sebuah permasalahan dimana didalam permasalahan tersebut terdapat sub-permasalahan yang bisa diselesaikan secara cepat. Fungsi rekursif sangat bergantung terhadap \textit{memory} komputer yang ada, semakin dalam fungsi tersebut semakin banyak \textit{memory} yang habis.

\begin{algorithm}
	\caption{SEQUENTIAL-SEARCH($L$, $i$, $j$, $x$)}
	\label{algo:seq-search}
	\begin{algorithmic}[1]
	\IF {$i \leq j$}
		\IF {L[i]==$x$}
			\RETURN $i$
		\ELSE
			\RETURN SEQUENTIAL-SEARCH($L$, $i$+1, $j$, $x$)
		\ENDIF
	\ELSE
		\RETURN 0
	\ENDIF	
	\end{algorithmic}
\end{algorithm}



\FloatBarrier
\begin{konsep}
\label{lat:fibo}
\textbf{Permasalahan perhitungan bilangan Fibonacci}\\
Hasilkan sebuah algoritma dan \textit{flowchart} untuk menentukan bilangan fibonacci urutan ke $n$.\\
\textbf{Masukan}\\
Sebuah bilangan bulat $n$.\\
\textbf{Keluaran}\\
Keluarannya berupa bilangan fibonacci ke $n$.\\
\begin{center}
\textbf{Test Case 1}\\
\end{center}
\textbf{Masukan}\\
3\\
\textbf{Keluaran}\\
2
\begin{center}
\textbf{Test Case 2}\\
\end{center}
\textbf{Masukan}\\
7\\
\textbf{Keluaran}\\
13
\end{konsep}







