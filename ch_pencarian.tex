\chapter{Teknik \textit{Pencarian}}\label{ch:modul4}

\section{Permasalahan Pencarian Minimum dan Maksimum}
Diberikan sebuah rangkaian bilangan, bagaimana cara paling cepat untuk mencari bilangan terbesar dan terkecil dari rangkaian tersebut? 

Untuk menyelesaikan masalah tersebut, ada beberapa pendekatan. Pendekatan yang naif adalah dengan melakukan \textit{looping} dari awal rangkaian sampai akhir rangkaian yang diperlihatkan di Algoritma \ref{algo:minmaxnaif}. Sedangkan untuk pendekatan \textit{Divide and Conguer} bisa dilihat di Algoritma \ref{algo:minmaxdac}.

\begin{algorithm}[H]
	\caption{MIN-MAX-NAIF($A$)}
	\label{algo:minmaxnaif}
	\begin{algorithmic}[1]
		\STATE $min = A[1]$
		\STATE $max = A[1]$
		\FOR{$i=2$ \TO $A.length$}
			\IF{$A[i] < min$}
				\STATE $min = A[i]$
			\ENDIF
			\IF{$A[i] > max$}
				\STATE $max = A[i]$
			\ENDIF
		\ENDFOR
		\RETURN $min,max$
	\end{algorithmic}
\end{algorithm}

Algoritma \ref{algo:minmaxnaif} memiliki jumlah komparasi sebanyak $2n-2$. Akan tetapi dengan menggunakan strategi \textit{divide and conquer} jumlah komparasi yang dilakukan bisa dikecilkan menjadi $(3n/2)-2$. Ini mungkin tidak signifikan, walaupun demikian untuk kasus lain strategi \textit{divide and conguer} bisa mengurangi jumlah instruksi secara drastis. 

Algoritma \ref{algo:minmaxdac} menggunakan strategi \textit{divide and conquer} untuk mencari nilai minimum dan maksimum dari rangkaian angka. Prinsipnya adalah dengan membagi rangkaian angka tersebut menjadi dua bagian yaitu $A[1..n/2]$ dan $A[(n/2)+2..n]$, mencari minimum dan maksimum di kedua bagian tersebut, dan terakhir mengambil minimum terkecil dari maksimum terbesar dari kedua bagian tersebut.

\begin{algorithm}[H]
	\caption{MIN-MAX-DAC($A$,$low$,$high$)}
	\label{algo:minmaxdac}
	\begin{algorithmic}[1]
		\IF{$high - low = 1$}
			\IF{$A[low] < A[high]$}
				\RETURN ($A[low]$,$A[high]$)
			\ELSE
				\RETURN ($A[high]$,$A[low]$)
			\ENDIF
		\ELSE
			\STATE $mid = \left\lfloor (low+high)/2\right\rfloor$
			\STATE $x_1,y_1$ = MIN-MAX-DAC($A$,$low$,$mid$)
			\STATE $x_2,y_2$ = MIN-MAX-DAC($A$,$mid+1$,$high$)
			\IF{$x_1 < x_2$}
				\STATE $x = x_1$
			\ELSE
				\STATE $x = x_2$
			\ENDIF
			\IF{$y_1 < y_2$}
				\STATE $y = y_1$
			\ELSE
				\STATE $y = y_2$
			\ENDIF  
			\RETURN $(x,y)$
		\ENDIF
	\end{algorithmic}
\end{algorithm}



\section{\textit{Binary Search}}

\textit{Binary Search} merupakan metode pencarian elemen di suatu rangkaian tertentu dengan menggunakan strategi \textit{divide and conquer}. Worst case \textit{binary search} adalah $O(log n)$ dimana jumlah komparasi terbanyak yang dilakukan adalah sebanyak $\left\lfloor log n \right\rfloor + 1$ untuk $n$ input.

\begin{algorithm}[H]
	\caption{BINARY-SEARCH($A,low,high,x$)}
	\begin{algorithmic}[1]
		\IF{$low > high$}
			\RETURN 0
		\ELSE
			\STATE $mid = \left\lfloor(low+high)/2)\right\rfloor$
			\IF{$x==A[mid]$}
				\RETURN $mid$
			\ELSIF{$x<A[mid]$}
				\RETURN BINARY-SEARCH($A,low,mid-1,x$) 
			\ELSE
				\RETURN BINARY-SEARCH($A,mid+1,high,x$)
			\ENDIF
		\ENDIF
	\end{algorithmic}
\end{algorithm}
