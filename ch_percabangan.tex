\chapter{Percabangan dan Perulangan (\textit{Branching and Looping})}

\section{Percabangan (\textit{Banching})}
\newthought{Percabangan} dibutuhkan ketika terdapat beberapa kemungkinan keputusan yang mungkin dalam alur program. Setiap kemungkinan tersebut bergantung terhadap nilai variabel ataupun berdasarkan hasil evaluasi \textbf{kondisi logika}. 

Kondisi logika adalah perbandingan dua atau lebih variabel dengan menggunakan operator logika sebagai berikut.
\begin{enumerate}
	\item Lebih besar ($>$)
	\item Lebih kecil ($<$)
	\item Lebih besar sama dengan ($\geq$)
	\item Lebih kecil sama dengan ($\leq$)
	\item Sama dengan ($==$)
	\item Tidak sama dengan (!= atau $<>$)
\end{enumerate}

Untuk menggabungkan beberapa kondisi logika digunakan \textbf{gerbang logika}\sidenote{Gerbang logika merupakan rangkaian dengan satu atau lebih sinyal masukan yang diproses untuk menghasilkan sinyal keluaran. Contoh gerbang logika adalah AND, NOT, OR, NOR, NAND dan XOR.}.\\
Hasil dari Percabangan dapat berupa dua nilai, Benar (\textit{True}) atau Salah (\textit{False}).

Contoh kondisi logika bisa dilihat di Contoh \ref{cth:kondisiLogika}.
\begin{contoh}
\label{cth:kondisiLogika}
\textbf{Kondisi Logika}\\

\begin{table}\index{typefaces!sizes}
	\centering
	\begin{tabular}{  l  c  }
	\hline
	Kondisi & Hasil \\
	\hline
	7 $>$ 3 & Salah \\
	8 $<$ 8 & Salah \\
	(5 $>$ 9) AND Benar & Salah \\
	(10 $<$ 2) OR (8 $\leq$ 8) & Benar \\
	NOT (3 $<$ 7) & Salah \\
	\hline
	\end{tabular}
\label{table:tabellogika}
\end{table}
\end{contoh}


\subsection{Struktur percabangan}
Percabangan biasanya menggunakan pernyataan \textbf{jika} (\textit{if}). Struktur pernyataan \textit{if} dalam bentuk pseudocode dapat dilihat sebagai berikut.

\begin{tabbing}
~~~~~\=\textbf{if} kondisi 1 \textbf{then}~~~~~~~~~~~~~~~\=\#Pengujian kondisi\\
\>~~~$<$\textbf{statements 1}$>$ \> \#Jika True jalankan\\
\>\textbf{else if } kondisi 2 \textbf{then}\>\#Opsional\\
\>~~~$<$\textbf{statements 2}$>$\>\\
\>\textbf{else}\>\#Opsional\\
\>~~~$<$\textbf{statements 3}$>$\>\\
\>\textbf{end if}
\end{tabbing}

\FloatBarrier
Contoh penggunaan statement \textit{IF} adalah sebagai berikut.

\begin{contoh}
	\textbf{Penggunaan IF..THEN..ELSE}
	\begin{algorithm}[H]
		\caption{PENENTUAN-NILAI()}
		\begin{algorithmic}[1]
		\IF{$x > 4$} 
			\STATE $y = 5$ \COMMENT{Perintah ini hanya dijalankan jika nilai $x > 4$.}
		\ELSIF{$x > 2$}
			\STATE $y = 8$ \COMMENT{Perintah ini hanya dijalankan jika nilai $2 < x < 4$.}
		\ELSE
			\STATE $y = 6$ \COMMENT{Perintah ini hanya dijalankan jika nilai $x \leq 4$.}
		\ENDIF
		\end{algorithmic}
	\end{algorithm}
\end{contoh}


Dalam bentuk flowchart, bisa dilihat di Gambar \ref{fig:flowchart-IF}.

\begin{marginfigure}%
\includegraphics[scale=0.6]{fig/flowchart-IF.eps}%
\caption{Flowchart dari \textit{IF Statement}}%
\label{fig:flowchart-IF}%
\end{marginfigure}

Format \textit{IF Statement} dalam Python bisa ditulis sebagai berikut.

\begin{tabbing}
~~~~~\=\textbf{if} $<$test1$>$:~~~~~~~~~~~~~~~\=\#Pengujian kondisi\\
\>~~~$<$statements1$>$ \> \#Jika True jalankan\\
\>\textbf{elif} $<$test2$>$:\>\#Opsional\\
\>~~~$<$statements2$>$\>\\
\>\textbf{else}:\>\#Opsional\\
\>~~~$<$statements3$>$\>\\
\end{tabbing}

Contoh program Python untuk mengecek bilangan genap atau ganjil bisa dilihat di Listing \ref{lst:genapDanGanjil}.

\begin{listprog}{genapDanGanjil.py}
	\label{lst:genapDanGanjil}
	\begin{lstlisting}[language=Python]
		num = input("Masukkan sebuah angka")
		if num%2 == 0:
				print num, " adalah bilangan genap."
		else:
				print num, " adalah bilangan ganjil."
	\end{lstlisting}
\end{listprog}


\subsection{Percabangan Bersarang (\textit{Nested branching})}
Sebuah percabangan dapat memiliki percabangan di dalamnya dan percabangan yang di dalam tersebut juga dapat memiliki percabangan lainnya di dalam. Sturuktur percabangan bersarang dapat dilihat pada pseudocode berikut.

\begin{tabbing}
~~~~~\=\textbf{if} kondisi 1 \textbf{then}~~~~~~~~~~~~~~~\=\#Pengujian kondisi\\
\>~~~\textbf{if} kondisi 1a \textbf{then}~~~~~~~~~~~~~~~\=\#Pengujian kondisi\\
\>~~~~~~$<$\textbf{statements}$>$ \> \#Jika True jalankan\\
\>~~~\textbf{else}\>\#Opsional\\
\>~~~~~~$<$\textbf{statements}$>$\>\\
\>~~~\textbf{end if}\\
\>\textbf{else if } kondisi 2 \textbf{then}\>\#Opsional\\
\>~~~$<$\textbf{statements 2}$>$\>\\
\>\textbf{else}\>\#Opsional\\
\>~~~$<$\textbf{statements 3}$>$\>\\
\>\textbf{end if}
\end{tabbing}

\section{Perulangan}
Perulangan dibutuhkan bilamana kita ingin mengeksekusi perintah yang sama berkali-kali dengan nilai yang berbeda maupun sama. Perulangan sering digunakan pada pemrosesan terhadap sekumpulan data, misalnya \textit{array}, \textit{list}, \textit{string} dan sebagainya.

\subsection{Struktur Perulangan}
Perulangan biasanya menggunakan pernyataan (\textit{for}) atau (\textit{while}). Pernyataan \textit{for} digunakan untuk mengiterasi sebuah \textit{array}, \textit{list} ataupun kumpulan variabel/objek lainnya sedangkan pernyataan \textit{while} digunakan untuk perulangan yang berdasarkan kondisi tertentu.

Struktur dari \textit{for} dalam pseudocode adalah sebagai berikut.
\begin{tabbing}
\textbf{for} $i=n$ \textbf{to} $m$~~~~~~~~~~~~~~~\=\#Mengisi variabel i, dan lakukan perulangan sebanyak (m-n-1)\\
~~~~~statements\\
\textbf{end for}
\end{tabbing}

\begin{contoh}
	\textbf{Penggunaan FOR}
	\begin{algorithm}
	\caption{PERULANGAN-FOR-CETAK-1-SAMPAI-5()}
		\begin{algorithmic}[1]
		\FOR{$i=1$ \TO $5$}
			\STATE print $i$
		\ENDFOR
		\STATE\COMMENT{Maka yang dicetak adalah 1 2 3 4 5}
		\STATE\COMMENT{Nilai $i$ terakhir yang tidak dicetak adalah 6}
		\end{algorithmic}
	\end{algorithm}
\end{contoh}

Dalam bahasa Python format \textit{for} adalah sebagai berikut.
\begin{tabbing}
\textbf{for} $i$ \textbf{in} $x$:~~~~~~~~~~~~~~~\=\#$x$ adalah \textit{kumpulan variabel}, iterasi dilakukan sebanyak panjang $x$\\
~~~~~statements\\
\end{tabbing}

Contoh penggunaan format Python untuk iterasi isi dari List bisa dilihat di Listing \ref{lst:iterasiArray}.
\begin{listprog}{iterasiList.py}
	\label{lst:iterasiArray}
	\begin{lstlisting}[language=Python]
		A = [4,1,3,5]
		for i in A:
			print i
		#Hasil print berupa 4 1 3 5
	\end{lstlisting}
\end{listprog}

Sedangkan untuk mencetak rangkaian bilangan misalnya dari 1 sampai 10 bisa menggunakan fungsi \textit{range}. Contohnya bisa dilihat di Listing \ref{lst:cetakBilangan}
\begin{listprog}{cetakBilangan.py}
	\label{lst:cetakBilangan}
	\begin{lstlisting}[language=Python]
		for i in range(1,11):
			print i
		#Hasil print berupa 1 2 3 4 5 6 7 8 9 10
	\end{lstlisting}
\end{listprog}

Untuk \textit{flowchart} \textit{for} bisa dilihat di Gambar \ref{fig:flowchartFor}.
\begin{figure}%
\centering
\includegraphics[scale=0.6]{fig/flowchart-FOR.eps}%
\caption{Flowchart For}%
\label{fig:flowchartFor}%
\end{figure}

\FloatBarrier
Untuk struktur \textit{while} bisa dilihat sebagai berikut.
\begin{tabbing}
\textbf{while} (Kondisi Logika)~~~~~~~~~~~~~~~\=\#Menjalankan perulangan selama kondisi benar\\
~~~~~statements\\
\textbf{end while}
\end{tabbing}

\FloatBarrier
\begin{contoh}
	\textbf{Penggunaan WHILE}
		\begin{algorithm}[H]
		\caption{PERULANGAN-WHILE-CETAK-1-SAMPAI-5()}
			\begin{algorithmic}[1]
			\STATE $i=1$
			\WHILE{$i<=5$}
				\STATE print $i$
				\STATE $i=i+1$
			\ENDWHILE
			\end{algorithmic}
		\end{algorithm}
\end{contoh}

Format bahasa Python untuk \textit{while} adalah sebagai berikut.

\begin{tabbing}
\textbf{while} (Kondisi Logika):~~~~~~~~~~~~~~~\=\#Menjalankan perulangan selama kondisi benar\\
~~~~~statements\\
\end{tabbing}

Contoh penggunaan \textit{while} dalam bahasa Python untuk mencetak menurun bilang 10 sampai 1 bisa dilihat di Listing \ref{lst:cetakBilanganTurun}.
\begin{listprog}{cetakBilanganTurun.py}
	\label{lst:cetakBilanganTurun}
	\begin{lstlisting}[language=Python]
		i = 10
		while(i>0):
			print i
			i = i - 1	\end
		#Hasil print berupa 10 9 8 7 6 5 4 3 2 1
	\end{lstlisting}
\end{listprog}


Untuk \textit{flowchart} \textit{while} bisa dilihat di Gambar \ref{fig:flowchartWhile}.
\begin{figure}%
\centering
\includegraphics[scale=0.6]{fig/flowchart-WHILE.eps}%
\caption{Flowchart While}%
\label{fig:flowchartWhile}%
\end{figure}


\FloatBarrier
\subsection{Perulangan Bersarang}
Sebuah perulangan dapat memiliki perulangan di dalamnya dan perulangan yang di dalam tersebut juga dapat juga dapat memiliki perulangan lainnya di dalam. Perulangan yang di dalam tersebut tidaklah harus menggunakan pernyataan yang sama dengan perulangan induknya. Misalnya, boleh saja kita menggunakan pernyataan perulangan \textit{for} di dalam pernyataan perulangan \textit{while} dan juga sebaliknya. Algoritma berikut akan menunjukkan salah satu struktur perulangan bersarang.

\begin{tabbing}
\textbf{for} $i=n$ \textbf{to} $m$~~~~~~~~~~~~~~~\=\#Mengisi variabel i, dan lakukan perulangan sebanyak (m-n-1)\\
~~~~~\textbf{for} $j=n$ \textbf{to} $m$~~~~~~~~~~~~~~~\=\#Mengisi variabel j, dan lakukan perulangan sebanyak (m-n-1)\\
~~~~~statements\\
~~~~~\textbf{end for}\\
\textbf{end for}
\end{tabbing}

	
 
\section{Gabungan Percabangan dan Perulangan}
Dalam kondisi tertentu, suatu perulangan dapat memiliki percabangan di dalamnya. Biasanya percabangan di dalam perulangan digunakan untuk mengkhususkan perintah berbeda yang akan dikerjakan ketika variabel mencapai nilai tertentu. Pada perulangan yang memiliki percabangan di dalamnya, mungkin saja terdapat perintah :
\begin{enumerate}
	\item \textit{break}, yang akan menghentikan perulangan walaupun nilai varibelnya belum melampaui batas.
	\item \textit{continue}, yang akan melanjutkan perulangan ke tahapan perulangan berikutnya dan tidak akan menjalankan perintah di bawahnya.
\end{enumerate}
Dua contoh berikut akan menunjukkannya.

\begin{contoh}
	\textbf{Penggunaan BREAK}
	\begin{algorithm}
	\caption{PERULANGAN-CETAK-ANGKA-DENGAN-BREAK()}
		\begin{algorithmic}[1]
		\FOR{$i=1$ \TO $5$}
			\IF {$i == 4$}
				\STATE \textbf{break}
			\ENDIF
			\STATE print $i$
		\ENDFOR
		\STATE\COMMENT{Maka yang dicetak adalah 1 2 3.}
		\STATE\COMMENT{Pada saat $i$ mencapai nilai 4, perulangan langsung berhenti sebelum sempat mencetak.}
		\end{algorithmic}
	\end{algorithm}
\end{contoh}

\FloatBarrier
\begin{contoh}
	\textbf{Penggunaan CONTINUE}
	\begin{algorithm}
	\caption{PERULANGAN-CETAK-ANGKA-DENGAN-CONTINUE()}
		\begin{algorithmic}[1]
		\FOR{$i=1$ \TO $5$}
			\IF {$i == 4$}
				\STATE \textbf{continue}
			\ENDIF
			\STATE print $i$
		\ENDFOR
		\STATE\COMMENT{Maka yang dicetak adalah 1 2 3 5.}
		\STATE\COMMENT{Perulangan tetap akan dilanjutkan hingga i = 5, tanpa mencetak nilai 4.}
		\end{algorithmic}
	\end{algorithm}
\end{contoh}

\FloatBarrier
\begin{konsep}
\label{lat:pencetakanMatriks}
\textbf{Permasalahan pencetakan matriks}
Hasilkan sebuah algoritma dan \textit{flowchart} untuk mencetak matriks dengan ketentuan:
\begin{enumerate}
	\item Baris ganjil dari 1 sampai n
	\item Baris genap dari n turun sampai 1
\end{enumerate}
\textbf{Masukan}\\
Sebuah bilangan bulat $n$.\\ 
\textbf{Keluaran}\\
Keluarannya berupa matriks $n$x$n$ dengan baris ganjil merupakan rangkaian angka menaik dari 1 sampai n dan baris genap merupakan rangkaian angka menurun dari n sampai 1.\\
\begin{center}
\textbf{Test Case 1}\\
\end{center}
\textbf{Masukan}\\
5\\
\textbf{Keluaran}\\
1 2 3 4 5 \\
5 4 3 2 1 \\
1 2 3 4 5 \\
5 4 3 2 1 \\
1 2 3 4 5 \\
\begin{center}
\textbf{Test Case 2}\\
\end{center}
\textbf{Masukan}\\
3\\
\textbf{Keluaran}\\
1 2 3 \\
3 2 1 \\
1 2 3 \\
\end{konsep}

\begin{pemrograman}
Buatkan program python dari Latihan \ref{lat:pencetakanMatriks}.
\end{pemrograman}

\begin{konsep}
\label{lat:pencetakanBintang}
\textbf{Permasalahan pencetakkan bintang dari NIM}\\
Hasilkan algoritma dan \textit{flowchart} untuk mencetak bintang dan angka berikut dengan menggunakan NIM sebagai dasarnya.\\
\textbf{Masukan}\\
Sebuah \textit{String} 9 karakter dimana merupakan NIM dari mahasiswa STMIK Mikroskil.\\
\textbf{Keluaran}\\
9 baris dimana di baris i terdapat karakter i, satu spasi dan diikuti bintang sejumlah karakter i tersebut.\\
\begin{center}
\textbf{Test Case 1}\\
\end{center}
\textbf{Masukan}\\
121114567\\
\textbf{Keluaran}\\
1 * \\
2 ** \\
1 * \\
1 * \\
1 * \\
0 \\
5 ***** \\
6 ****** \\
7 ******* \\
\begin{center}
\textbf{Test Case 2}\\
\end{center}
\textbf{Masukan}\\
031110023\\
\textbf{Keluaran}\\
0  \\
3 *** \\
1 * \\
1 * \\
1 * \\
0 \\
0 \\
2 ** \\
3 *** \\
\end{konsep}

\begin{pemrograman}
Buatkan program python dari Latihan \ref{lat:pencetakanBintang}.
\end{pemrograman}